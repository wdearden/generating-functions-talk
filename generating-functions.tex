\documentclass{beamer}
\usetheme{Copenhagen}
\usepackage[utf8]{inputenc}
\usepackage{amsmath}
 
 
%Information to be included in the title page:
\title{Generating Functions}
\author{Will Dearden}
\institute{Uptake}
\date{2019-02-08}
 
 
 
\begin{document}
 
\frame{\titlepage}
 
\begin{frame}
    \frametitle{What is a generating function?}
    Let $a = a_1, a_2, \ldots $ be a sequence.

    \begin{block}{Definition}
        The generating function for a sequence $a$ is $$G_a(x) = a_0 + a_1 \cdot x + a_2 \cdot x^2 + \cdots$$
    \end{block}
\end{frame}

\begin{frame}
    \frametitle{Why do we care about generating functions?}
    Working with discrete things is tough.
    \pause
    \newline
    \newline
    Generating functions allow us to use:
    \begin{itemize}
        \item Algebra
        \item Calculus
        \item Complex Analysis
    \end{itemize}

    \pause
    \quad
    \newline
    I will cover the "calculus of generating functions" and how basic operations on generating functions correspond to operations on the sequence itself.
\end{frame}

\begin{frame}
    \frametitle{The Fibonacci numbers}

    The Fibonacci numbers are defined by a simple recurrence.
    $$F_{n + 1} = F_n + F_{n-1} \quad \left(n \geq 1;F_0 = 0;F_1 = 1\right)$$

    \pause

    You might have been taught a closed-form formula for the $n$th Fibonacci number. Maybe you even proved it using induction.

    \pause
    We're going to discover it using generating functions.
\end{frame}

\begin{frame}
    \frametitle{The Fibonacci numbers}

    The generating function for $F$ is $$G_F(x) = F_0 + F_1 \cdot x + F_2 \cdot x^2 + \cdots$$

    \pause

    Let's multiply it by $x$

    \pause

    \begin{align*}
        x \cdot G_F(x) &= F_0\cdot x + F_1 \cdot x^2 + F_2 \cdot x^3 + \cdots \\
        &= F_1 \cdot x^2 + F_2 \cdot x^3 + \cdots 
    \end{align*}

    \pause

    Now by $x^2$

    \pause

    $$x^2 \cdot G_F(x) = F_0\cdot x^2 + F_1 \cdot x^3 + F_2 \cdot x^4 + \cdots$$
\end{frame}

\begin{frame}
    \frametitle{The Fibonacci numbers}

    Now we add them together

    \pause

    \begin{align*}
        x \cdot G_F(x) + x^2 \cdot G_F(x) &= \quad F_1 \cdot x^2 + F_2 \cdot x^3 + \cdots \\
        &\quad + F_0\cdot x^2 + F_1 \cdot x^3 + F_2 \cdot x^4 + \cdots \\
        &= (F_1 + F_0) \cdot x^2 + (F_2 + F_1) \cdot x^3 + \cdots \\
        &= F_2 \cdot x^2 + F_3 \cdot x^3 + \cdots \\
        &= G_F(x) - x
    \end{align*}

    \pause

    Solve that equation for $G_F$ and we get

    $$G_F(x) = \frac{x}{1 - x - x^2}$$
\end{frame}

\begin{frame}
    \frametitle{The Fibonacci numbers}

    We want to simplify $$\frac{x}{1 - x - x^2}$$ \pause

    Now we're in the realm of algebra. What can we simplify it to? \pause

    You might remember $$\frac{1}{1 - y} = 1 + y + y^2 + y^3 + \cdots $$ \pause

    In other words $$\frac{1}{1 - x}$$ is the generating function for $a = 1, 1, 1, \ldots $.

\end{frame}

\begin{frame}
    \frametitle{The Fibonacci numbers}

    So we want the partial fraction decomposition for $$\frac{x}{1 - x - x^2}$$ \pause

    We can use the quadratic formula to factor $1 - x - x^2$.

    $$1 - x - x^2 = (1 - \phi_+ x)(1 - \phi_- x)$$ $$ \phi_{\pm} = \frac{1 \pm \sqrt{5}}{2} \approx 1.618, -0.618$$

\end{frame}

\begin{frame}
    \frametitle{The Fibonacci numbers}

    If we do a little bit of algebra we then get

    $$1 - x - x^2 = \frac{1}{\sqrt{5}} \left( \frac{1}{1 - \phi_+ x} - \frac{1}{1 - \phi_- x }\right)$$ \pause

    But we can use the fact that $\frac{1}{1 - y} = 1 + y + y^2 \cdots $ to expand out each of those fractions

    $$\frac{1}{\sqrt{5}} \left(\left(1 + \phi_+ \cdot x + \phi_+^2 \cdot x^2 + \cdots \right) - \left(1 + \phi_- \cdot x + \phi_-^2 \cdot x^2 + \cdots \right)\right)$$

\end{frame}

\begin{frame}
    \frametitle{The Fibonacci numbers}

    Now we have an exact formula for the coefficients of $x_n$ so we know $F_n$.

    $$F_n = \frac{1}{\sqrt{5}} \left( \phi_+^n - \phi_-^n\right)$$ \pause

    Since $\phi_-^n$ gets vanishingly small we can also say

    $$F_n \approx 0.447 \cdot 1.618^n$$


\end{frame}

\begin{frame}
    \frametitle{The Calculus of Generating Functions}

    We used a lot of little tricks there. I'm going to list out a bunch of tricks we can use to work with generating functions.

\end{frame}

\begin{frame}
    \frametitle{The Calculus of Generating Functions}

    \begin{block}{Generating functions preserve linearity}
        $G_{A + B} = G_A + G_B$ \newline
        $G_{c \cdot A} = c \cdot G_A$ 
    \end{block}

    \pause

    \begin{block}{Multiplying by $x$ shifts the sequence}
        $x \cdot G_{[A_{n+1}]} = G_A - A_0$
    \end{block}

    \begin{align*}
        x \cdot G_{[A_{n+1}]} &= x \cdot (A_1 + A_2 \cdot x + A_3 \cdot x^2 + \cdots) \\
        &= A_1 \cdot x + A_2 \cdot x^2 + A_3 \cdot x^3 + \cdots \\
        &= G_A - A_0
    \end{align*}

\end{frame}

\begin{frame}
    \frametitle{The Calculus of Generating Functions}

    \begin{block}{Taking the derivative multiplies by $n$}
        $x \cdot G'_A(x) = G_{[n*A_n]}$
    \end{block}

    \begin{align*}
        x \cdot G'_A(x) &= x \cdot \frac{d(A_0 + A_1 \cdot x + A_2 \cdot x^2 + A_3 \cdot x^3 + \cdots )}{dx} \\
        &= x \cdot (A_1 + 2A_2 \cdot x + 3A_3 \cdot x^2 + \cdots ) \\
        &= A_1 \cdot x + 2A_2 \cdot x^2 + 3A_3 \cdot x^3 + \cdots \\
        &= G_{[n*A_n]}
    \end{align*}

\end{frame}

\begin{frame}
    \frametitle{The Calculus of Generating Functions}

    Let's define the convolution of sequence. For two sequence $A$ and $B$, $A \ast B$ is defined as

    $$(A \ast B)_n = \sum_{i + j = n} A_i \cdot B_j$$ Then we get another rule.

\end{frame}

\begin{frame}
    \frametitle{The Calculus of Generating Functions}

    \begin{block}{Multiplying generating functions takes a convolution}
        $G_A \cdot G_B = G_{A \ast B}$
    \end{block}

    \begin{align*}
        G_A \cdot G_B &= (A_0 + A_1 \cdot x + B_2 \cdot x^2 + \cdots) \\
        & \quad \cdot (B_0 + B_1 \cdot x + B_2 \cdot x^2 + \cdots) \\
        &= (A_0 + B_0) + (A_1 \cdot B_0 + A_0 \cdot B_1) \cdot x \\
        & \quad + (A_2 \cdot B_0 + A_1 \cdot B_1 + A_0 \cdot B_2) \cdot x^2 \\
        &= G_{A \ast B}
    \end{align*} \pause

    Let $B = 1, 1, 1, \ldots$. One cool thing to note is that implies $\frac{G_A}{1 - x}$ is the generating function for the partial sums of $A$.

\end{frame}

\begin{frame}
    \frametitle{Another example}

    Note that the previous rule applies to multiplying more than two generating functions together. 
    \pause
    \quad
    \newline
    \newline
    Let's try raising a simple generating function to a power. What is $\frac{1}{(1-x)^k}$ the generating function of?
    \pause
    \quad
    \newline
    \newline
    $\frac{1}{1 - x}$ is the generating function for $[1]$.
    \pause
    \quad
    \newline
    \newline
    Then $\frac{1}{(1-x)^k}$ is the generating function for $$\sum_{n_1 + n_2 + \cdots + n_k = n} 1 \cdot 1 \cdots 1$$
    \pause
    In other words $\frac{1}{(1-x)^k}$ is the generating function for the number of ways to have an ordered sequence of length $k$ of nonnegative integers add up to $n$.

\end{frame}

\begin{frame}
    \frametitle{Another example}

    Let's look at all the ways to have 3 nonnegative integers add up to 2. We have 

    \begin{align*}
        2 &= 2 + 0 + 0 \\
        &= 1 + 1 + 0 \\
        &= 1 + 0 + 0 \\
        &= 0 + 2 + 0 \\
        &= 0 + 1 + 1 \\
        &= 0 + 0 + 2
    \end{align*}

\end{frame}

\begin{frame}
    \frametitle{Another example}

    It turns out this is equivalent to allocating $k-1$ plus signs among $n$ ones. Let the $\star$ stand for a 1.

    \begin{align*}
        2 = 2 + 0 + 0 &\Leftrightarrow \star \star + +\\
        = 1 + 1 + 0 &\Leftrightarrow \star + \star +\\
        = 1 + 0 + 1 &\Leftrightarrow \star + + \star\\
        = 0 + 2 + 0 &\Leftrightarrow + \star \star + \\
        = 0 + 1 + 1 &\Leftrightarrow + \star + \star \\
        = 0 + 0 + 2 &\Leftrightarrow + + \star \star
    \end{align*}

\end{frame}

\begin{frame}
    \frametitle{Another example}

    So we have $n + k - 1$ slots and we need to choose where to put $k-1$ plus signs. The number of ways to do this is ${n + k - 1 \choose k - 1}$. \pause
    \quad
    \newline
    \newline
    And just to get this to something simpler it turns out the generating function for ${n \choose k}$ is $\frac{x ^ k}{(1 - x)^{k + 1}}$.

\end{frame}

\begin{frame}
    \frametitle{Another example}

    What is $\sum_{k=1}^{n} k^2$?
    \pause
    \quad
    \newline
    \newline
    Generating function of $[n^2]$ is $\frac{x(1 + x)}{(1 - x)^3}$ by taking the derivative of $\frac{1}{1 - x}$ twice.
    \pause
    \quad
    \newline
    \newline
    Divide by $1 - x$ to get the generating function of $\sum_{k=1}^{n} k^2$ is $$\frac{x(1 + x)}{(1 - x)^4} = \frac{x^2}{(1 - x)^4} + \frac{x}{(1 - x)^4}$$
    \pause
    Since dividing by $x$ is equivalent to shifting by $n$ and we have the combinatorial generating function from before, we get formula
    $$\sum_{k=1}^{n} k^2 = {n + 1 \choose 3} + {n + 2 \choose 3} = \frac{n(n + 1)(2n + 1)}{6}$$.


\end{frame}

\begin{frame}
    \frametitle{Coin counting}

    Your turn: let's say we have pennies, nickels, dimes, and quarters. What's the generating function for the number of ways to make change for $n$ cents?

\end{frame}
 
\end{document}